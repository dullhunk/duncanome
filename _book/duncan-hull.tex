\documentclass[12pt,]{book}
\usepackage{lmodern}
\usepackage{amssymb,amsmath}
\usepackage{ifxetex,ifluatex}
\usepackage{fixltx2e} % provides \textsubscript
\ifnum 0\ifxetex 1\fi\ifluatex 1\fi=0 % if pdftex
  \usepackage[T1]{fontenc}
  \usepackage[utf8]{inputenc}
\else % if luatex or xelatex
  \ifxetex
    \usepackage{mathspec}
  \else
    \usepackage{fontspec}
  \fi
  \defaultfontfeatures{Ligatures=TeX,Scale=MatchLowercase}
\fi
% use upquote if available, for straight quotes in verbatim environments
\IfFileExists{upquote.sty}{\usepackage{upquote}}{}
% use microtype if available
\IfFileExists{microtype.sty}{%
\usepackage{microtype}
\UseMicrotypeSet[protrusion]{basicmath} % disable protrusion for tt fonts
}{}
\usepackage{hyperref}
\hypersetup{unicode=true,
            pdftitle={Dr.~Duncan Hull},
            pdfborder={0 0 0},
            breaklinks=true}
\urlstyle{same}  % don't use monospace font for urls
\usepackage{natbib}
\bibliographystyle{apalike}
\usepackage{longtable,booktabs}
\usepackage{graphicx,grffile}
\makeatletter
\def\maxwidth{\ifdim\Gin@nat@width>\linewidth\linewidth\else\Gin@nat@width\fi}
\def\maxheight{\ifdim\Gin@nat@height>\textheight\textheight\else\Gin@nat@height\fi}
\makeatother
% Scale images if necessary, so that they will not overflow the page
% margins by default, and it is still possible to overwrite the defaults
% using explicit options in \includegraphics[width, height, ...]{}
\setkeys{Gin}{width=\maxwidth,height=\maxheight,keepaspectratio}
\usepackage[normalem]{ulem}
% avoid problems with \sout in headers with hyperref:
\pdfstringdefDisableCommands{\renewcommand{\sout}{}}
\IfFileExists{parskip.sty}{%
\usepackage{parskip}
}{% else
\setlength{\parindent}{0pt}
\setlength{\parskip}{6pt plus 2pt minus 1pt}
}
\setlength{\emergencystretch}{3em}  % prevent overfull lines
\providecommand{\tightlist}{%
  \setlength{\itemsep}{0pt}\setlength{\parskip}{0pt}}
\setcounter{secnumdepth}{5}
% Redefines (sub)paragraphs to behave more like sections
\ifx\paragraph\undefined\else
\let\oldparagraph\paragraph
\renewcommand{\paragraph}[1]{\oldparagraph{#1}\mbox{}}
\fi
\ifx\subparagraph\undefined\else
\let\oldsubparagraph\subparagraph
\renewcommand{\subparagraph}[1]{\oldsubparagraph{#1}\mbox{}}
\fi

%%% Use protect on footnotes to avoid problems with footnotes in titles
\let\rmarkdownfootnote\footnote%
\def\footnote{\protect\rmarkdownfootnote}

%%% Change title format to be more compact
\usepackage{titling}

% Create subtitle command for use in maketitle
\providecommand{\subtitle}[1]{
  \posttitle{
    \begin{center}\large#1\end{center}
    }
}

\setlength{\droptitle}{-2em}

  \title{Dr.~Duncan Hull}
    \pretitle{\vspace{\droptitle}\centering\huge}
  \posttitle{\par}
    \author{}
    \preauthor{}\postauthor{}
    \date{}
    \predate{}\postdate{}
  
\usepackage{booktabs}

\begin{document}
\maketitle

{
\setcounter{tocdepth}{1}
\tableofcontents
}
\hypertarget{about}{%
\chapter*{About}\label{about}}
\addcontentsline{toc}{chapter}{About}

\begin{center}\includegraphics[width=0.4\linewidth]{images/me-blue} \end{center}

Hello, my name is Duncan. I'm a lecturer in the \href{https://www.cs.manchester.ac.uk/}{Department of Computer Science} at the \href{https://www.manchester.ac.uk}{University of Manchester} where I lead the \href{https://www.cs.manchester.ac.uk/study/undergraduate/industrial-experience/}{Industrial Experience (IE)} program. This elective course has over 100 students every year working for 12 months in industry in the penultimate year of their degree. 🎓

I \protect\hyperlink{teaching}{teach} undergraduate courses, supervise tutorials, final year projects and masters projects. I serve as \href{http://studentnet.cs.manchester.ac.uk/ugt/year2/}{second year tutor}, employability tutor, while also serving on the \href{http://www.studentsupport.manchester.ac.uk/study-support/mitigating-circumstances/}{mitigating circumstances} committee and the exam board. I'm interested in methods that can \protect\hyperlink{research}{deliver high quality learning and student experience} by using innovative techniques like \protect\hyperlink{vertical-tutoring-1}{vertical tutoring}, \href{https://www.cs.manchester.ac.uk/connect/business-engagement/industrial-mentoring/}{industrial mentoring}, \href{https://en.wikipedia.org/wiki/Live_coding}{live coding}, \protect\hyperlink{wikipedia}{Wikipedia editing} and live / recorded music. 🎸

If you are an \protect\hyperlink{employers}{employer} who would like to recruit a summer intern, placement student or graduate please \protect\hyperlink{contact}{get in touch}. During term time, we highlight opportunities for students in the \href{https://waggle.cs.manchester.ac.uk/waggle/about}{Wednesday Waggle}. 🐝

\hypertarget{background}{%
\section*{Background}\label{background}}
\addcontentsline{toc}{section}{Background}

My \href{https://uk.linkedin.com/in/duncanhull}{background} is a mixture of Natural Sciences (\href{http://www.plantsciences.manchester.ac.uk/}{Plant Sciences, BSc}), Computer Science (\href{http://www.cs.man.ac.uk/~hulld/msc2003.html}{MSc} \& PhD) and software engineering. I've worked as a consultant and software developer for various organisations including \href{https://en.wikipedia.org/wiki/BBC_Monitoring}{BBC Monitoring}, the \href{https://en.wikipedia.org/wiki/Ford_Motor_Company}{Ford Motor Company} and the \href{https://en.wikipedia.org/wiki/National_Health_Service}{National Health Service} (NHS). While working on \href{https://en.wikipedia.org/wiki/Apache_Taverna}{Apache Taverna} and \href{https://en.wikipedia.org/wiki/MyGrid}{myGrid} I completed a \href{https://ethos.bl.uk/OrderDetails.do?uin=uk.bl.ethos.497578}{PhD} at the University of Manchester. This was followed by a \href{https://en.wikipedia.org/wiki/Postdoctoral_researcher}{postdoc} at the \href{http://www.mib.ac.uk/}{Manchester Institute of Biotechnology} (MIB) on the \href{http://www.nactem.ac.uk/pathtext/}{Refine project} and a stint as a software engineer on \href{https://www.ebi.ac.uk/chebi/}{Chemical Entities of Biological Interest} (ChEBI) in Cambridge, UK at the European Bioinformatics Institute (\href{https://www.ebi.ac.uk}{ebi.ac.uk}). 🇪🇺

\hypertarget{full-stack-teaching}{%
\section*{Full stack teaching}\label{full-stack-teaching}}
\addcontentsline{toc}{section}{Full stack teaching}

I have taught english, maths, science and engineering at a range of levels from primary through to postgraduate. In 2011, I completed a \href{https://en.wikipedia.org/wiki/Postgraduate_Certificate_in_Education}{PGCE} at the \href{https://www.bath.ac.uk/}{University of Bath} and trained at co-educational \href{https://en.wikipedia.org/wiki/State-funded_schools_(England)}{non-selective state-funded schools} in \href{https://www.lydiardparkacademy.org.uk/}{Swindon}, \href{https://shaftesburyschool.co.uk/}{Shaftesbury} and \href{https://www.st-annes.stockport.sch.uk/}{Stockport} before returning to higher education. Regardless of the age or the stage, I enjoy the challenges of teaching and have taught primary \& secondary school children (\href{https://en.wikipedia.org/wiki/K\%E2\%80\%9312}{K--12}), undergraduates \& postgraduates in the UK, India and America. 🇬🇧🇮🇳🇺🇸

\hypertarget{tools}{%
\section*{Tools}\label{tools}}
\addcontentsline{toc}{section}{Tools}

This website is written in \href{https://rmarkdown.rstudio.com/}{R markdown} and built using \href{https://bookdown.org}{bookdown}, \href{https://www.gitbook.com}{gitbook}, \href{http://www.jabref.org/}{JabRef}, \href{https://en.wikipedia.org/wiki/JavaScript}{JavaScript}, \href{https://en.wikipedia.org/wiki/Knitr}{knitr}, \href{https://en.wikipedia.org/wiki/LaTeX}{LaTeX}, \href{https://pandoc.org/}{Pandoc}, \href{https://www.rstudio.com/}{RStudio} and \href{https://code.visualstudio.com/}{Visual Studio Code}. Thanks to \href{https://yihui.name/}{Yihui Xie} for the excellent tools and documentation. The \href{https://github.com/dullhunk/hulled}{source is available on github}, but you'll be better off reading the manual \emph{\href{https://bookdown.org/yihui/bookdown/}{Authoring Books and Technical Documents with R Markdown}} first. If you're reading this on an iPhone or iPad, there is \href{https://github.com/rstudio/bookdown/issues/60}{known bug with the menu bar at the top of this page} which means the menu might not display properly. I could have (should have?) used \href{https://bookdown.org/yihui/blogdown/}{blogdown} and \href{https://gohugo.io}{Hugo}, but opted for bookdown because it is much \sout{less bloated} easier to use. 🔨

\hypertarget{teaching}{%
\chapter{Students}\label{teaching}}

I teach, mentor, tutor, lecture on and supervise a variety of undergraduate and postgraduate courses. You can find me in the labs, my office and the lecture theatre. 🎭

\textbackslash{}begin\{figure\}

\{\centering \includegraphics[width=0.75\linewidth]{images/question-everything}

\}

\textbackslash{}caption\{Question everything, or \emph{\href{https://en.wikipedia.org/wiki/Nullius_in_verba}{Nullius in verba}} as the say at the Royal Society\}\label{fig:unnamed-chunk-2}
\textbackslash{}end\{figure\}

\hypertarget{all-years-debug-your-cv}{%
\section{All years: debug your CV}\label{all-years-debug-your-cv}}

\begin{itemize}
\tightlist
\item
  You can drop-in to my weekly one-to-one CV clinics for Computer Science students in LF25 during term-time.
\item
  If you haven't written a CV (two pages), résumé (one page) or LinkedIn profile before, you might find the \emph{Debug your CV} guide useful at \href{http://git.io/mycv}{git.io/mycv}.
\item
  Get feedback on your CV from as many people as possible, because ``\href{https://en.wikipedia.org/wiki/Linus\%27s_Law}{given enough eyeballs, all bugs are shallow}'' \citep{Raymond1999}
\item
  Outside of term time, it's best to book an appointment
\end{itemize}

\hypertarget{first-year}{%
\section{First year}\label{first-year}}

\begin{itemize}
\tightlist
\item
  Teaching on \href{https://studentnet.cs.manchester.ac.uk/ugt/COMP10120/syllabus/}{First year team projects: COMP101} led by \href{http://www.cs.man.ac.uk/~sattler/}{Ulrike Sattler}
\item
  Mentoring one group of six first year students
\item
  Organising first year guest lectures, which mostly run in the second semester, February to May
\end{itemize}

\hypertarget{second-year}{%
\section{Second year}\label{second-year}}

\begin{itemize}
\tightlist
\item
  Teaching on \href{https://studentnet.cs.manchester.ac.uk/ugt/COMP23311/syllabus/}{Second year software engineering: COMP23311} led by \href{http://www.cs.man.ac.uk/~embury/}{Suzanne Embury}
\item
  Organising the labs for the \href{https://www.cs.manchester.ac.uk/connect/business-engagement/industrial-mentoring/}{software engineering mentoring program}
\item
  Leading second year tutorials COMP2CARS which focus on wellbeing and working out your next steps.
\end{itemize}

\hypertarget{penultimate-year}{%
\section{Penultimate year}\label{penultimate-year}}

\begin{itemize}
\tightlist
\item
  Leading the course for ``with industrial experience'' (IE), an elective and intercalated year in industry.
\item
  Visiting students on placement
\end{itemize}

\hypertarget{final-year}{%
\section{Final year}\label{final-year}}

\begin{itemize}
\tightlist
\item
  Supervising final year educational projects based in secondary schools in Manchester, see \href{https://git.io/computinged}{git.io/computinged}.
\end{itemize}

\hypertarget{masters}{%
\section{Masters}\label{masters}}

\begin{itemize}
\tightlist
\item
  Supervising \href{https://www.cs.manchester.ac.uk/study/masters/}{Master of Science} projects in Computer Science and Data Science. This usually involves research using Wikipedia, Wikdata, \href{https://en.wikipedia.org/wiki/SPARQL}{SPARQL} and chatbots. 🤖
\end{itemize}

\hypertarget{extra-curricular}{%
\section{Extra-curricular}\label{extra-curricular}}

\begin{itemize}
\tightlist
\item
  Organising, facilitating and promoting extra-curricular activities, usually off-timetable (for example Wednesday afternoons, evenings and weekends). I'm proud to have served as a judge of the fantastic \href{https://www.studenthack.com}{studenthack.com} and \href{https://greatunihack.com}{greatunihack.com} since 2014. These \href{https://en.wikipedia.org/wiki/Hackathon}{hackathons}\citep{hafb} are organised by \href{https://www.unicsmcr.com/}{UniCS}, a student-led tech society formerly known as HackSoc and CSSoc.
\end{itemize}

\hypertarget{employers}{%
\chapter{Employers}\label{employers}}

We work with a wide range of employers from the smallest bedroom startup to the worlds largest multi-national corporations, and are always looking for more organisations that can offer our students a stimulating working environment. According to \href{https://www.highfliers.co.uk}{highfliers.co.uk}, the University of Manchester is the most targeted University in the UK by the \href{https://www.top100graduateemployers.com}{Times Top 100 Graduate Employers}. We can still do better, for example by engaging with a more diverse group of employers, especially those in Manchester and the \href{https://northernpowerhouse.gov.uk/}{Northern Powerhouse} (see \href{https://git.io/manc}{git.io/manc}).

\begin{center}\includegraphics[width=1\linewidth]{images/industry-club-wide} \end{center}

\hypertarget{recruiting-students}{%
\section{Recruiting students}\label{recruiting-students}}

If you are recruiting computer scientists and software engineers as a summer interns, placement students or as graduates please get in touch with me or \href{https://uk.linkedin.com/in/mabel-yau}{Mabel Yau} (careers and placements officer). We typically have around 250 undergraduate students graduating annually, alongside a smaller number of Masters and PhD students.

If you are looking to recruit students from related degree disciplines like \href{https://www.physics.manchester.ac.uk/}{Physics}, \href{https://www.maths.manchester.ac.uk/}{Maths}, \href{https://www.chemistry.manchester.ac.uk/}{Chemistry}, \href{https://www.mace.manchester.ac.uk/}{MACE}, \href{https://www.materials.manchester.ac.uk/}{Materials} and \href{https://www.eee.manchester.ac.uk/}{EEE} you should talk to the Careers Service centrally at \href{http://www.careers.manchester.ac.uk/}{careers.manchester.ac.uk}

\hypertarget{careers-fairs}{%
\section{Careers fairs}\label{careers-fairs}}

Our annual Computer Science careers fair is held in the Kilburn building in autumn, we typically have around 30 employers exhibiting over two days. As space is limited, we are always over-subscribed and are not able to accommodate every employer that our students will be interested in. The central careers service also organises:

\begin{itemize}
\tightlist
\item
  the big careers fair in \href{https://www.manchestercentral.co.uk/}{Manchester Central} every autumn, see the \href{http://www.careers.manchester.ac.uk/events/bigcareersfair/}{Big Careers Fair}
\item
  a smaller careers fair in Fallowfield \href{http://www.sport.manchester.ac.uk/facilities/armitage/}{Armitage centre} in May
\item
  hundreds of other employer events on campus during term time
\end{itemize}

\hypertarget{drop-in-sessions}{%
\section{Drop-in sessions}\label{drop-in-sessions}}

If you aren't able to exhibit at careers fairs, we also run ad-hoc drop-in sessions where employers can come in and set up a stand in the foyer to talk to computer science students informally on their way to and from lectures. These usually happen during lunch in \href{https://www.manchester.ac.uk/discover/key-dates/}{term time}. If you're interested in exhibiting at either of these events, please \protect\hyperlink{contact}{contact the careers and placements officer Mabel Yau}.

\hypertarget{industry-club}{%
\section{Industry Club}\label{industry-club}}

\begin{center}\includegraphics[width=0.4\linewidth]{images/industry-club-black} \end{center}

All employers are welcome to join our industry club mailing list by sending an email to \href{mailto:listserv@listserv.manchester.ac.uk}{\nolinkurl{listserv@listserv.manchester.ac.uk}} with the the text \textbf{subscribe cs-industryclub yourfirstname yoursecondname} in the body of the email message. The industry club is part of our \href{https://www.cs.manchester.ac.uk/connect/business-engagement/}{wider business engagement activities}.

The mailing list is low-traffic, typically two to three updates per year and an invitation to our annual industry club meeting. We promise not to spam you or sell your email details on to third parties.

\hypertarget{industrial-mentoring}{%
\section{Industrial mentoring}\label{industrial-mentoring}}

The \href{https://www.cs.manchester.ac.uk/connect/business-engagement/industrial-mentoring/}{Industrial mentoring scheme for software engineers} allows employers meet students during code review sessions.

\hypertarget{the-wednesday-waggle}{%
\section{The Wednesday Waggle}\label{the-wednesday-waggle}}

During term time, we highlight events and vacancies for Computer Science students from a \href{http://dullhunk.github.io/where-can-I-look-for-jobs.html}{wide range of sources} in the \href{https://waggle.cs.manchester.ac.uk/waggle/about}{Wednesday Waggle}. If you have vacancies or events you would like our students to know about, \protect\hyperlink{contact}{get in touch with us} or \href{http://www.careers.manchester.ac.uk/aboutus/contact/}{contact the careers service}.

\hypertarget{buzzin-our-kid}{%
\section{Buzzin' our kid! 🐝}\label{buzzin-our-kid}}

At peak times, we can get \textbf{very busy} with many concurrent employer events on campus. \citep{highfliers2019} Please be patient and persistent if we do not reply immediately. Unfortunately, we are not always able to respond to everyone because our students, staff and space are all finite resources. We give priority to employers that have already given their time and expertise to our community.

\begin{figure}

{\centering \includegraphics[width=1\linewidth]{images/high-fliers-table} 

}

\caption{According to [highfliers.co.uk](https://www.highfliers.co.uk), the University of Manchester is the most targeted University in the UK by the [Times Top 100 Graduate Employers](https://www.top100graduateemployers.com) [@highfliers2019]}\label{fig:unnamed-chunk-5}
\end{figure}

\hypertarget{research}{%
\chapter{Research}\label{research}}

My research interests are in \href{https://scholar.google.com/citations?view_op=search_authors\&hl=en\&mauthors=label:computer_science_education}{Computer Science Education} (CSE) and \href{https://en.wikipedia.org/wiki/Pedagogy}{pedagogy}. \citep{JohnBiggs2011} I'm interested in methods that can deliver high quality learning and student experience using innovative techniques like \protect\hyperlink{vertical-tutoring-1}{vertical tutoring}, \href{https://www.cs.manchester.ac.uk/connect/business-engagement/industrial-mentoring/}{industrial mentoring}, live music, \href{https://en.wikipedia.org/wiki/Live_coding}{live coding}, \protect\hyperlink{wikipedia}{Wikipedia editing} and more.

\textbackslash{}begin\{figure\}

\{\centering \includegraphics[width=0.7\linewidth]{images/wikipedian_protester}

\}

\textbackslash{}caption\{Too many educational practices are not backed up by good evidence that they actually work. More evidence is needed to support many of the claims made about effective pedagogy. \emph{Wikipedian Protester} cartoon by \href{https://en.wikipedia.org/wiki/Randall_Munroe}{Randall Munroe} at \href{https://xkcd.com/285/}{xkcd.com/285} Creative Commons Attribution-NonCommercial 2.5 License\}\label{fig:unnamed-chunk-6}
\textbackslash{}end\{figure\}

\hypertarget{sigcse}{%
\section{SIGCSE}\label{sigcse}}

Computer Science has only been taught to undergraduates in the UK \href{http://www.bbc.co.uk/manchester/content/articles/2005/11/07/baby_computer_40_interview_feature.shtml}{for 50 short years}, so there's lots of open questions about how to teach both the practical and theoretical aspects of the subject. To that end:

\begin{itemize}
\tightlist
\item
  I'm an active member of the \href{https://en.wikipedia.org/wiki/Association_for_Computing_Machinery}{Association for Computing Machinery} (ACM) Special Interest Group (SIG) in Computer Science Education (\href{https://sigcse.org}{SIGCSE.org}).
\item
  As part of that I founded and chair a \href{https://duncan.hull.name/2019/07/17/sigcse-journal-club/}{journal club} for educators in Manchester, if you'd like to join us, subscribe to the mailing list by emailing \href{mailto:listserv@listserv.manchester.ac.uk}{\nolinkurl{listserv@listserv.manchester.ac.uk}} with the text \textbf{subscribe sigcse-journal-club yourfirstname yoursecondname} in the body of your message
\item
  I'm serving on the program committee for \href{http://community.dur.ac.uk/cep.conference}{Computing Education \& Practice (CEP)} conference at Durham University in 2020.
\end{itemize}

\hypertarget{industrial-mentoring-1}{%
\section{Industrial mentoring}\label{industrial-mentoring-1}}

Since we started the \href{https://www.cs.manchester.ac.uk/connect/business-engagement/industrial-mentoring/}{Industrial mentoring scheme for software engineers} in 2015, more than 1000 students have been through the mentoring scheme with 250 students taking the course every year. We are very grateful for continued support from our industrial partners in making this happen.

\hypertarget{vertical-tutoring}{%
\section{Vertical tutoring}\label{vertical-tutoring}}

We are currently piloting a vertical tutoring (VT) scheme, see \protect\hyperlink{vertical-tutoring-1}{vertical tutoring} for details.

\hypertarget{codeclub}{%
\section{Code Club}\label{codeclub}}

I lead an after school \href{https://codeclub.org}{CodeClub} as part of a global network of free coding clubs for 9--13 year olds. \citep{codeclub} The aim is to have fun using scratch, python and other interesting technology we can get our hands on including \href{https://www.raspberrypi.org/}{Raspberry Pi}, \href{https://microbit.org/}{Micro:bits}, \href{https://www.lego.com/en-gb/themes/mindstorms}{LEGO® MINDSTORMS®}, \href{https://www.oculus.com}{Oculus Rift}, \href{https://sonic-pi.net/}{Sonic Pi} and \href{http://www.codebug.org.uk/}{CodeBug} etc.

\hypertarget{wikipedia}{%
\section{Wikipedia}\label{wikipedia}}

Wikipedia (and its sister project \href{https://www.wikidata.org}{wikidata.org}) \citep{Vrandecic2014} are powerful tools for improving digital skills and communication skills, regardless of your age or level of computer literacy. \citep{Proffitt2018} \citep{goodfaith} As an experienced and long serving editor of Wikipedia since 2004, I organise and participate in \href{https://en.wikipedia.org/wiki/Edit-a-thon}{edit-a-thons} which recruit and train new Wikipedia editors.\citep{goodbadugly} \citep{troubled} More information at:

\begin{itemize}
\tightlist
\item
  \href{https://wiki-loves-scientists.org.uk/}{wiki-loves-scientists.org.uk}
\item
  \href{https://en.wikipedia.org/wiki/User:Duncan.Hull}{en.wikipedia.org/wiki/User:Duncan.Hull}
\end{itemize}

\hypertarget{informal-publications}{%
\section{Informal publications}\label{informal-publications}}

Informal publications can be found on my sporadically updated blog

\begin{itemize}
\tightlist
\item
  \href{https://duncan.hull.name/lablog/}{duncan.hull.name/lablog}
\end{itemize}

\hypertarget{formal-publications}{%
\section{Formal publications}\label{formal-publications}}

Formal peer-reviewed publications can be found on DBLP, ORCID and Google Scholar\ldots{}

\begin{itemize}
\tightlist
\item
  \href{https://dblp.org/pid/h/DuncanHull}{dblp.org/pid/h/DuncanHull}
\item
  \href{https://orcid.org/0000-0003-2387-503X}{orcid.org/0000-0003-2387-503X}
\item
  \href{https://scholar.google.com/citations?user=iDJ-t7IAAAAJ}{scholar.google.com/citations?user=iDJ-t7IAAAAJ}
\end{itemize}

\ldots{}and even wikidata:

\begin{itemize}
\tightlist
\item
  \href{https://www.wikidata.org/wiki/Q47012855}{wikidata.org/wiki/Q47012855}
\end{itemize}

\hypertarget{vertical-tutoring-1}{%
\chapter{Vertical tutoring}\label{vertical-tutoring-1}}

We are currently piloting a vertical tutoring (VT) system for undergraduate students. VT is already widespread in secondary education (see \href{https://www.verticaltutoring.org/}{verticaltutoring.org}), \citep{vtbernard} but as far as we know has not been used in higher education.

Extending the idea of Peer Assisted Study Sessions (PASS) \href{http://www.pass.manchester.ac.uk}{pass.manchester.ac.uk}, vertical tutoring creates tutorial groups with a representative from \emph{one of each} year of undergraduate study combined with alumni.

\begin{figure}

{\centering \includegraphics[width=1\linewidth]{images/vertical-tutor-groups} 

}

\caption{Conventional horizontal tutor groups (shown in black) bring together a group of students in the same year. For example, year 1 students meet as a small group once per week during term time with their tutor. Vertical tutor groups (shown in red) are made of of one student from each year and an alumni. Vertical tutor groups extend the idea of PASS, to full stack mentoring, crossing all levels}\label{fig:unnamed-chunk-7}
\end{figure}

\hypertarget{full-stack-mentoring}{%
\section{Full stack mentoring}\label{full-stack-mentoring}}

A vertical tutor group will typically contain five members as shown in Figure\textasciitilde{}\ref{fig:vertical-fig}. The group meets physically where possible, or virtually via a slack channel which consists of:

\begin{enumerate}
\def\labelenumi{\arabic{enumi}.}
\tightlist
\item
  One first year student
\item
  One second year student
\item
  One penultimate year student (out on industrial placement)
\item
  One final year student (returned from placement or summer internship)
\item
  One member of our alumni, recent graduate or via \href{https://www.network.manchester.ac.uk/}{network.manchester.ac.uk}
\end{enumerate}

Vertical tutor groups meet twice per semester. It is very unlikely that a free timetable slot for all years and alumni can be found during normal office hours, because of the complexities of timetabling. So evenings will be likely to work best. Where possible, tutor groups will meet face to face, with remote members (e.g.~placement students and alumni) typically joining virtually by slack or similar.

\hypertarget{what-is-it-good-for}{%
\section{What is it good for?}\label{what-is-it-good-for}}

Vertical tutoring is an attractive idea but does it actually work? If so, how? What is it useful for? We would like to find out:

\begin{enumerate}
\def\labelenumi{\arabic{enumi}.}
\tightlist
\item
  If there is any appetite for vertical tutoring amongst students and alumni
\item
  How it could work e.g.~with \href{https://slack.com}{slack.com} or \href{https://discordapp.com/}{discord} etc?
\item
  How many times can/should vertical tutor groups meet? Twice per semester? More frequently? Less frequently?
\item
  What are suitable topics for discussion in a vertical tutorial? Careers, wellbeing, networking etc
\item
  What kind of specialist groups could be useful e.g.~all female groups, research focussed tutorial groups (with MSc \& PhD students), ordinary ``vanilla'' groups etc
\end{enumerate}

As this is an experiment, students have been selected on a voluntary basis. If you're a student or former student and would like to get involved, let me know.

\hypertarget{how-long-will-all-this-take}{%
\section{How long will all this take?}\label{how-long-will-all-this-take}}

We ask that tutees commit to:

\begin{itemize}
\tightlist
\item
  two one hour sessions per semester
\item
  some setup and administration, slack channels, scheduling suitable times and dates with your group
\item
  Two hours of time for feedback and review after each semester, by email survey
\end{itemize}

\hypertarget{coding-their-future}{%
\chapter{Coding their future}\label{coding-their-future}}

\textbf{Coding their future} is a collaboration \& partnership between your school and the \href{https://www.cs.manchester.ac.uk/}{Department of Computer Science} at the University of Manchester to improve and support Computer Science education at key stages 3 and 4. \citep{afterthereboot} \citep{cambridgegcse}

\begin{center}\includegraphics[width=1\linewidth]{images/schools-banner} \end{center}

We provide schools with a final year student who can teach Computer Science in your school or college as a teaching assistant. In return, the school provides our undergraduate students with a safe and supportive environment in which to teach which extends and augments your curriculum. This can either be an after school, extension / lunchtime club or during scheduled lesson time, typically between year 7 and 13. Since these projects were started in 2012, our students have worked with a range of schools in the private and public sector, both selective and non-selective, co-educational and single-sex including:

\begin{itemize}
\tightlist
\item
  \href{http://www.fairfieldhigh.tameside.sch.uk/}{Fairfield High School for Girls}, Droylsden
\item
  \href{http://www.utcmediacityuk.org.uk/}{University Technical College (UTC) \(@\)MediaCityUK}, Salford
\item
  \href{https://www.manchestercommunicationacademy.com/}{Manchester Communication Academy}, Harpurhey
\item
  \href{https://thebarlowrchigh.co.uk/}{The Barlow RC High School}, Didsbury
\item
  \href{https://www.chhs.org.uk/}{Cheadle Hulme High School}, Stockport
\item
  \href{http://www.aggs.trafford.sch.uk/}{Altrincham Grammar School for Girls}, Trafford
\item
  \href{https://www.agsb.co.uk/}{Altrincham Grammar School for Boys}, Trafford
\item
  \href{https://www.mgs.org/}{Manchester Grammar School}, Fallowfield
\end{itemize}

The projects were setup by \protect\hyperlink{contact}{Duncan Hull} and \href{http://www.cs.man.ac.uk/~david/}{David Rydeheard} (retired 2019), and are now run and supervised by Duncan. We hope to transfer ideas between private and public sector, to find out more, see \protect\hyperlink{guidance-for-teachers}{guidance for teachers} and \protect\hyperlink{guidance-for-students}{guidance for students} below.

\hypertarget{guidance-for-teachers}{%
\section{Guidance for teachers}\label{guidance-for-teachers}}

Our aim is to support the teaching and learning of Computer Science in your school and to help engage schoolchildren in the subject. This page describes what we can provide you with and what we expect to get in return.

\hypertarget{what-the-university-is-offering-your-school}{%
\subsection{What the University is offering your school}\label{what-the-university-is-offering-your-school}}

The University of Manchester will provide your school or college with at least one student ambassador with some relevant training who has completed two years of study in Computer Science and has:

\begin{itemize}
\tightlist
\item
  A good knowledge of, and enthusiasm for Computer Science
\item
  Completed DBS clearance
\item
  An interest in teaching and working with young people
\item
  Achieved a minimum of a 2:1 or 1st class degree in their second year
\end{itemize}

\hypertarget{what-the-university-expects-from-your-school}{%
\subsection{What the University expects from your school}\label{what-the-university-expects-from-your-school}}

In return, we expect that the school provides the undergraduate student with:

\begin{itemize}
\tightlist
\item
  Opportunities to engage with a classroom or after school club of children as a Teaching Assistant (TA). This is typically for around one or two hours during term time. Initially, this could be through classroom observation and teacher assistance, culminating in the student delivering at least one lesson (and potentially a series of lessons) with your support and guidance
\item
  Advice, suggestions, feedback, assessment and encouragement from you to suggest the kinds of resources that would be useful, appropriate or engaging for the Computer Science curriculum you are teaching
\item
  Classroom and behaviour management: the students are not trained teachers and will be relying on your expertise in classroom and behaviour management.
\end{itemize}

\hypertarget{resources-developed-by-students}{%
\subsection{Resources developed by students}\label{resources-developed-by-students}}

Undergraduates typically develop a range of resources. The project will involve development of a computer-based system together with supporting activities, lessons and resources. The resource could be a variety of things including, a game, robotics, animations, hardware (Raspberry Pi, Arduino etc) or software, intended to enthuse school students at one of the Key stages 3 or 4 about fundamental concepts in computing preferably linked to one of the new Computer Science curricula.

\hypertarget{project-timing}{%
\subsection{Project timing}\label{project-timing}}

The projects run for 6 months from September to March, divided into three phases.

\begin{enumerate}
\def\labelenumi{\arabic{enumi}.}
\tightlist
\item
  \textbf{September to October} Observation in the classroom teaching by the student around once per week. Development of ideas for an educational tool that the student will make, with the advice of the classroom teacher
\item
  \textbf{November to January} From November to January, our students develop and tests prototype tool (or tools) with the supporting material, this can happen sooner for students who make a quick start to the project.
\item
  \textbf{February to April} From February to April, our students are expected to liaise closely with teachers to develop an educational tool that will be of use in the classroom using teachers' suggestions as to what is appropriate to build. Students will spend some time in a classroom working closely with teachers and students developing and delivering a new resource for teaching. More details on final year projects can be found in COMP300, the undergraduates already know what is required from their project
\end{enumerate}

\hypertarget{assessment-and-monitoring}{%
\subsection{Assessment and monitoring}\label{assessment-and-monitoring}}

Formal supervision and mentoring is undertaken by the university (Duncan and David), but we will ask you to fill in a one page form on your assessment of their progress during their time at your school, we very much value your input and hope that these projects can beneficial for both your school and the University. We don't want to burden you with unnecessary bureaucracy that all teachers battle with!

\hypertarget{guidance-for-students}{%
\section{Guidance for students}\label{guidance-for-students}}

So why would you, an undergraduate student, want to work on an education project in secondary school? The UK government would like Computer Science should be taught in all secondary schools in the UK.\citep{afterthereboot} However, in many UK schools there is a shortage of teachers who are trained in Computer Science, consequently, many teachers find themselves being asked to teach a subject they may know little about.\citep{shutdownrestart}

Undergraduate students can make a significant difference here, by supporting teachers in the classroom to create and deliver new classroom resources in Computer Science.\citep{computinged} In addition, you will be able to:

\begin{itemize}
\tightlist
\item
  develop leadership skills in the classroom
\item
  gain valuable experience of working on ``real world'' problems in a stimulating environment
\item
  improve your communication skills, especially spoken communication
  work as part of a team (in the school) and join a small group of like-minded undergraduate students (in the University) working on related projects
\item
  test your knowledge \& technical ability in a challenging and dynamic environment working with young people
\item
  last, but not least, there is a good chance you will have lots of fun and have a rewarding experience of teaching
  make yourself more employable by doing all of the above
\end{itemize}

\hypertarget{who-is-involved}{%
\subsection{Who is involved?}\label{who-is-involved}}

Initially, the number of undergraduate students involved in these projects will be less than ten. We also require that you will have a minimum of a 2:1 or 1st in your second year exams. Projects are supervised by Dr.~Duncan Hull with additional supervision from an experienced member of teaching staff at a participating school.

We have carefully selected schools in Manchester that are relatively easy for you to get to, are already teaching Computer Science and have supportive staff and teachers in place to help you. You will be expected to work directly with school children with the support of the teaching staff in your school. Schools we have worked with are all the Manchester area.

\hypertarget{what-will-the-educational-projects-be-expected-to-deliver}{%
\subsection{What will the educational projects be expected to deliver?}\label{what-will-the-educational-projects-be-expected-to-deliver}}

You will be expected to work closely with the teacher to develop resources that

\begin{itemize}
\tightlist
\item
  engage students with one or more aspects of the new Computer Science curriculum at an appropriate key stage. This is usually key stage 3 and/or 4, ages 11-16.
\item
  complement the schools current provision for computer science in the school
\end{itemize}

In order to achieve this, you will need to attend a series of education and outreach events, as part of the programme and sign up and complete the Science, Technology, Engineering \& Mathematics (STEMnet) ambassador training (which includes DBS checks) supplied by the Museum of Science and Industry (MOSI) in Manchester.

During the project you will be spending a significant amount of time in the classroom, visiting your school at least once every two weeks throughout the duration of your project to develop resources. These must include a computer-based teaching tool which may use, for example, Raspberry Pi's, visual aids, demonstrations, videos, online questionnaires, formative feedback, games, drones, robotics etc. In addition, guidance on classroom use, such as a lesson or series of lessons to support the tool.

All deliverables for standard final year projects will be expected of these projects including:

\begin{itemize}
\tightlist
\item
  first semester presentation
\item
  demonstration of the resource being used in the classroom
\item
  final written report
\end{itemize}

Assessments for these projects will be as for standard projects, but part of the evaluation of the project will be a classroom demonstration, a description and evaluation of which should be included in your final report.

\hypertarget{when-do-the-projects-start-and-finish}{%
\subsection{When do the projects start and finish?}\label{when-do-the-projects-start-and-finish}}

Projects start in September and are handed at Easter time, see final year project guidelines. For more information contact \protect\hyperlink{contact}{Duncan Hull}.

\hypertarget{contact}{%
\chapter{Contact}\label{contact}}

You can contact us using the the details below

\begin{center}\includegraphics[width=1\linewidth]{images/beescape} \end{center}

\hypertarget{office}{%
\section{Office}\label{office}}

Our offices are in the Kilburn building, close to the Byte cafe, past the \href{https://studentnet.cs.manchester.ac.uk/student-services/}{Student Support Office} (SSO), through the double doors, down the ramp.

\textbf{Dr.~Duncan Hull, Lecturer} 👨‍💻

\begin{itemize}
\tightlist
\item
  🏢 Room LF25, Kilburn Building
\item
  📥 email: duncan.hull ATE manchester.ac.uk
\item
  ☎️ telephone: +44 161 275 6186
\item
  🌐 \href{https://uk.linkedin.com/in/duncanhull}{linkedin.com/in/duncanhull}
\end{itemize}

\textbf{Mabel Yau, Careers and placements officer} 👩‍💻

\begin{itemize}
\tightlist
\item
  🏢 Room LF26, Kilburn Building
\item
  📥 email: mabel.yau ATE manchester.ac.uk
\item
  ☎️ telephone: +44 161 275 6140
\item
  🌐 \href{https://uk.linkedin.com/in/mabel-yau}{linkedin.com/in/mabel-yau}
\end{itemize}

\textbf{Student Support Office } 👨‍👩‍👧‍👦

\begin{itemize}
\tightlist
\item
  🏢 Room LF21, Kilburn Building
\item
  📥 email \href{mailto:compsci-sso@manchester.ac.uk}{\nolinkurl{compsci-sso@manchester.ac.uk}}
\item
  ☎️ telephone: +44 161 306 8155
\end{itemize}

\hypertarget{elsewhere}{%
\section{Elsewhere}\label{elsewhere}}

You can get in touch via t'internet at:

\begin{itemize}
\tightlist
\item
  Slack: search for ``Duncan Hull'' or my work email
\item
  Skype: search for ``duncanhull''
\item
  Blog: \href{https://duncan.hull.name}{duncan.hull.name}
\item
  Github: \href{https://github.com/dullhunk}{github.com/dullhunk}
\item
  Twitter: \href{https://twitter.com/dullhunk}{twitter.com/dullhunk}
\end{itemize}

\hypertarget{postal-address}{%
\section{Postal address}\label{postal-address}}

Send post 🐌 to :

Dr.~Duncan Hull, Lecturer
Department of Computer Science, Kilburn Building
The University of Manchester
Oxford Road
Manchester, M13 9PL

\hypertarget{kilburn-building-directions}{%
\section{Kilburn building directions}\label{kilburn-building-directions}}

From the train stations, it takes about 20 minutes to walk from \href{https://www.nationalrail.co.uk/stations_destinations/man.aspx}{Manchester Piccadilly} (MAN) and ten minutes from \href{https://www.nationalrail.co.uk/stations/mco/details.aspx}{Manchester Oxford Road} (MCO). Our official postcode (M13 9PL) takes you to \href{http://www.conference.manchester.ac.uk/venues/search/details/?property=10}{University Place} next door, so you're better of using the \href{https://www.bbc.co.uk/news/uk-england-49319760}{what3words locations} \citep{what3words} below which are more accurate:

\begin{itemize}
\tightlist
\item
  Google map of the Kilburn building \href{http://bit.ly/directions-to-kilburn-building}{bit.ly/directions-to-kilburn-building}
\item
  There are two ground floor entrances to the Kilburn building, North and South

  \begin{itemize}
  \tightlist
  \item
    North entrance: \href{https://what3words.com/port.museum.rips}{what3words.com/port.museum.rips}
  \item
    South entrance: \href{https://what3words.com/common.wiping.email}{what3words.com/common.wiping.email}
  \end{itemize}
\item
  There is no formal reception so the best place to meet is \href{http://bit.ly/ByteCafe}{bit.ly/ByteCafe} on the first floor
\item
  See also \href{https://www.cs.manchester.ac.uk/about/maps-and-travel/}{cs.manchester.ac.uk/about/maps-and-travel/}
\end{itemize}

\hypertarget{parking}{%
\section{Parking}\label{parking}}

If you are driving, the nearest car parks are:

\begin{itemize}
\tightlist
\item
  \textbf{University Car Park B} \href{https://www.ncp.co.uk/find-a-car-park/car-parks/manchester-aquatic-centre-jv/}{Manchester Aquatics Centre Car Park}, NCP \href{http://maps.google.co.uk/maps?q=M13+9SS}{M13 9SS}
\item
  \textbf{University Car Park D} Booth Street West Car Park, \href{http://maps.google.co.uk/maps?q=M15+6AR}{M15 6AR}, access via Higher Cambridge Street
\item
  See \href{https://www.estates.manchester.ac.uk/services/operationalservices/carparking/}{estates.manchester.ac.uk/services/operationalservices/carparking}
  .
\end{itemize}

\bibliography{hulld.bib}


\end{document}
